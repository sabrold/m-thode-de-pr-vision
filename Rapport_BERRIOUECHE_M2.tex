\documentclass[a4paper,12pt]{report}
\usepackage[utf8]{inputenc}
\usepackage{graphicx}
\usepackage[T1]{fontenc}
\usepackage[french]{babel}
\usepackage{tikz}
\usepackage{color}
\usepackage{soul}
\usepackage{caption}
\usepackage{amsmath}
\usepackage{amssymb}
\usepackage{ifthen}
\usepackage{fancybox}
\usepackage{fancyhdr}
\usepackage{geometry}
\usepackage{}
\geometry{left=1cm , right=1cm , top=1.5 cm , bottom=1.5cm}
\thispagestyle{empty}
\begin{document}


\begin{center}
\textbf{République Algérienne Démocratique et Populaire \\
\setlength{\parskip}{2mm}
Ministère de l'enseignement Supérieure et de la Recherche Scientifique\\ }
\setlength{\parskip}{3mm}
\textbf{Université des sciences et de la Technologie Houari Boumediene \\
\setlength{\parskip}{3mm} }
\setlength{\parskip}{3mm}
\begin{Huge}
\setlength{\parskip}{4mm}
\begin{center}
\includegraphics[width=5cm]{logo_usthb.png}
\end{center}

\textbf{Faculté de Mathématique}
\end{Huge}
\end{center}

\begin{center}
\textbf{Département de Recherche Opérationnelle}
\end{center}
\vspace{1\baselineskip}

\begin{center}
\begin{Huge}
\textbf{Rapport}

\end{Huge}\\
\end{center}

\bigskip
\begin{center}
\begin{Large}
\textbf{Méthodes de Prévision \\
\setlength{\parskip}{3mm}
Lissage exponentiel }
\end{Large}
\end{center}
\vspace{5\baselineskip}

\begin{itemize}
\item[•]\textbf{Présenté par:}
\setlength{\parskip}{3mm}

\begin{itemize} \item[-] BERRIOUECHE Sabrina
\end{itemize}
\setlength{\parskip}{3mm}
\item[•]\textbf{Professeur du module:}
\setlength{\parskip}{3mm}

\begin{itemize} \item[-] Mr CHAABANE Djamal
\end{itemize}


\end{itemize}
\vspace{5\baselineskip}
\begin{center}
\textbf{2021 - 2022}
\end{center}
\begin{document}

\part{SUR LE PLAN TH\'{E}ORIQUE}

\subsection{\textbf{Lissage exponentiel simple} :}

\qquad  Avec l'absence de la tendance et de la saisonnalit\'{e}.\\

\begin{itemize}
\item [•]\underline{\textbf{Algorithme}} :

\_\_\_\_\_\_\_\_\_\_\_\_\_\_\_\_\_\_\_\_\_\_\_\_\_\_\_\_\_\_\_\_\_\_\_\_\_\_%
\_\_\_\_\_\_\_\_\_\_\_\_\_\_\_\_\_\_\_ 

\'{E}tape 01: (Initialisation)

\texttt{\qquad \qquad La valeur initiale du lissage est fix\'{e}e par }

\texttt{\qquad \qquad \qquad }$\widehat{\mathtt{x}}_{1}=x_{1}\ ou\ \widehat{%
\mathtt{x}}_{1}=\overline{\mathtt{x}}$

\texttt{\qquad \qquad Lire }$\alpha $

\'{E}tape 02:

\texttt{\qquad \qquad Pour j = 2 \`{a} N}

\texttt{\qquad \qquad \qquad on calcule }$\widehat{\mathtt{x}}_{1}=\widehat{%
\mathtt{x}}_{j-1}\mathtt{+(1-\alpha )(x}_{j}-\widehat{\mathtt{x}}_{j-1})$

\texttt{\qquad \qquad Fin pour;}

\_\_\_\_\_\_\_\_\_\_\_\_\_\_\_\_\_\_\_\_\_\_\_\_\_\_\_\_\_\_\_\_\_\_\_\_\_\_%
\_\_\_\_\_\_\_\_\_\_\_\_\_\_\_\_\_\_\_
\\

\item[•] \underline{\textbf{Justification de l'algorithme}} :\\

La valeur $\widehat{X}_{N}(h)=$ fournie par le mod\`{e}le du lissage
exponentiel simple avec la constante de lissage $\alpha $ (entre $0$ et $1)$
est 
\begin{equation}
\widehat{X}_{N}(h)=\widehat{X}_{N}+h=(1-\alpha )\cdot \sum_{j=0}^{N-1}\alpha
^{j}X_{N-j}
\end{equation}

On remplace $N$ par $N-1$ dans $(1)$%
\begin{equation*}
\widehat{X}_{N-1}(h)=\widehat{X}_{N-1}+h=(1-\alpha )\cdot
\sum_{j=0}^{N-1}\alpha ^{j}X_{N-1-j}
\end{equation*}%
On multiplie par $\alpha $%
\begin{equation*}
\alpha \cdot \widehat{X}_{N-1}(h)=\widehat{X}_{N-1}+h=(1-\alpha )\cdot
\sum_{j=0}^{N-2}\alpha ^{j+1}X_{N-(j+1)}
\end{equation*}%
\begin{equation}
\alpha \cdot \widehat{X}_{N-1}(h)=(1-\alpha )\sum_{j=0}^{N-2}\alpha
^{j^{^{\prime }}}X_{N-j^{\prime }}
\end{equation}%
On fait la soustraction de $(1)-(2)$, on aura%
\begin{eqnarray*}
\widehat{X}_{N}-\alpha \cdot \widehat{X}_{N-1}(h) &=&(1-\alpha )\cdot
\sum_{j=0}^{N-1}\alpha ^{j}X_{N-j}^{\prime }-(1-\alpha
)\sum_{j=0}^{N-2}\alpha ^{j^{^{\prime }}}X_{N-j^{\prime }} \\
&=&(1-\alpha )\cdot (\sum_{j=0}^{N-1}\alpha ^{j}X_{N-j}^{\prime
}-\sum_{j=0}^{N-2}\alpha ^{j^{^{\prime }}}X_{N-j^{\prime }}) \\
&=&(1-\alpha )\cdot X_{N}
\end{eqnarray*}%
Qui donne la formule de lissage simple suivante : 
\begin{equation*}
\widehat{X}_{N}=\widehat{X}_{N-1}+(1-\alpha )(X_{N}-\widehat{X}_{N-1})
\end{equation*}%
Et la valeur de premier lissage $\widehat{X}_{1}=X_{1}.$
\end{itemize}

\subsection{\textbf{Lissage exponentiel double} :}

\qquad  Avec la pr\'{e}sence de la tendance . \\

\begin{itemize}
\item[•] \underline{\textbf{Algorithme}} :

\_\_\_\_\_\_\_\_\_\_\_\_\_\_\_\_\_\_\_\_\_\_\_\_\_\_\_\_\_\_\_\_\_\_\_\_\_\_%
\_\_\_\_\_\_\_\_\_\_\_\_\_\_\_\_\_\_\_

\'{E}tape 01: (Initialisation)

\qquad \qquad $S_{1}(1)=X$ et $S_{2}(1)=(1-\alpha )^{2}\cdot X_{1}$

\'{E}tape 02:

\qquad \qquad Pour $j=2$ \`{a} $N$

\qquad \qquad \qquad $S_{1}(j)=\alpha \cdot S_{1}(j-1)+(1-\alpha )\cdot
X_{j} $

\qquad \qquad \qquad $S_{2}(j)=\alpha \cdot S_{2}(j-1)+(1-\alpha )\cdot
S_{1}(j)$

\qquad \qquad Fin pour;

\'{E}tape 03:

\qquad \qquad On calcule

\qquad \qquad \qquad $\widehat{\alpha }_{1}(N)=2S_{1}(N)-S_{2}(N)$

\qquad \qquad \qquad $\widehat{\alpha }_{2}(N)=\frac{(1-\alpha )}{\alpha }%
(S_{1}(N)-S_{2}(N))$

\'{E}tape 04:

\qquad \qquad Pr\'{e}vision

\qquad \qquad \qquad $\widehat{X}_{N+h}=\widehat{\alpha }_{1}(N)+\widehat{%
\alpha }_{2}(N)+h$
\end{itemize}

\_\_\_\_\_\_\_\_\_\_\_\_\_\_\_\_\_\_\_\_\_\_\_\_\_\_\_\_\_\_\_\_\_\_\_\_\_\_%
\_\_\_\_\_\_\_\_\_\_\_\_\_\_\_\_\_\_\_
\\

\begin{itemize}
\item[•] \underline{\textbf{Justification}} :\\

Le mod\`{e}le de lissage double est donn\'{e} par la formule suivante :%
\begin{equation*}
\widehat{X}_{N+h}=\widehat{\alpha }_{1}(N)+\widehat{\alpha }_{2}(N)+h
\end{equation*}

On cherche $\widehat{\alpha }_{1}(N)$ et $\widehat{\alpha }_{2}(N)$ par 
\begin{equation*}
\min_{\widehat{\alpha }_{1},\widehat{\alpha }_{2}}S^{2}=||\alpha (X_{t}-%
\widehat{\alpha }_{1}-\widehat{\alpha }_{2}(t-N))||^{2}\Longrightarrow
Somme=\sum_{j=0}^{N-1}\alpha ^{j}(X_{N-j}-\widehat{\alpha }_{1}+\widehat{%
\alpha }_{2}\times j)^{2}
\end{equation*}%
On d\'{e}rive la formule "somme" par rapport \`{a} $\widehat{\alpha }_{1}$
et $\widehat{\alpha }_{2},$ on ontient 
\begin{equation*}
\left\{ 
\begin{array}{c}
\frac{\partial S^{2}}{\partial \widehat{\alpha }_{1}}=\frac{\partial }{%
\partial \widehat{\alpha }_{1}}(\sum_{j=0}^{N-1}\alpha ^{j}(X_{N-j}-\widehat{%
\alpha }_{1}+\widehat{\alpha }_{2}\times j)^{2})=-2\sum_{j=0}^{N-1}\alpha
^{j}(X_{N-j}-\widehat{\alpha }_{1}+\widehat{\alpha }_{2}\times j)^{2} \\ 
\\ 
\frac{\partial S^{2}}{\partial \widehat{\alpha }_{2}}=\frac{\partial }{%
\partial \widehat{\alpha }_{2}}(\sum_{j=0}^{N-1}\alpha ^{j}(X_{N-j}-\widehat{%
\alpha }_{1}+\widehat{\alpha }_{2}\times j)^{2})=-2\sum_{j=0}^{N-1}j\alpha
^{j}(X_{N-j}-\widehat{\alpha }_{1}+\widehat{\alpha }_{2}\times j)^{2}%
\end{array}%
\right. 
\end{equation*}%
On s'annule les deux d\'{e}riv\'{e}es, on aura%
\begin{equation*}
\left\{ 
\begin{array}{c}
\frac{\partial S^{2}}{\partial \widehat{\alpha }_{1}}=0 \\ 
\\ 
\frac{\partial S^{2}}{\partial \widehat{\alpha }_{2}}=0%
\end{array}%
\right. \Longrightarrow \left\{ 
\begin{array}{c}
-2\sum_{j=0}^{N-1}\alpha ^{j}(X_{N-j}-\widehat{\alpha }_{1}+\widehat{\alpha }%
_{2}\times j)^{2}=0 \\ 
\\ 
-2\sum_{j=0}^{N-1}j\alpha ^{j}(X_{N-j}-\widehat{\alpha }_{1}+\widehat{\alpha 
}_{2}\times j)^{2}=0%
\end{array}%
\right. 
\end{equation*}

\begin{equation}
\Longrightarrow \left\{ 
\begin{array}{c}
\widehat{\alpha }_{1}\sum_{j=0}^{N-1}\alpha ^{j}-\widehat{\alpha }%
_{2}\sum_{j=0}^{N-1}j\alpha ^{j}=\sum_{j=0}^{N-1}\alpha ^{j}X_{N-j}\text{ \
\ } \\ 
\\ 
\widehat{\alpha }_{1}\sum_{j=0}^{N-1}j\alpha ^{j}-\widehat{\alpha }%
_{2}\sum_{j=0}^{N-1}j^{2}\alpha ^{j}=\sum_{j=0}^{N-1}j\alpha ^{j}X_{N-j}%
\end{array}%
\right. 
\end{equation}

Quand $N$ tend vers $\infty ,$ on aura 
\begin{eqnarray*}
\sum_{j=0}^{N-1}\alpha ^{j} &=&\frac{1}{1-\alpha } \\
\sum_{j=0}^{N-1}j\alpha ^{j} &=&\frac{\alpha }{(1-\alpha )^{2}} \\
\sum_{j=0}^{N-1}j^{2}\alpha ^{j} &=&\frac{\alpha (\alpha +1)}{(1-\alpha )^{3}%
}
\end{eqnarray*}%
On remplace $(3)$%
\begin{equation*}
\left\{ 
\begin{array}{c}
\widehat{\alpha }_{1}-\widehat{\alpha }_{2}\frac{1}{1-\alpha }=(1-\alpha
)\sum_{j=0}^{N-1}\alpha ^{j}X_{N-j} \\ 
\\ 
\widehat{\alpha }_{1}\alpha -\widehat{\alpha }_{2}\frac{\alpha (\alpha +1)}{%
(1-\alpha )}=(1-\alpha )^{2}\sum_{j=0}^{N-1}j\alpha ^{j}X_{N-j}%
\end{array}%
\right. 
\end{equation*}

On d\'{e}finie les lissage comme suit 
\begin{equation*}
\left\{ 
\begin{array}{c}
S_{1}(N)=(1-\alpha )\sum_{j=0}^{N-1}\alpha ^{j}X_{N-j} \\ 
\\ 
S_{1}(N)=(1-\alpha )\sum_{j=0}^{N-1}\alpha ^{j}S_{1}(N-j)%
\end{array}%
\right. \Rightarrow \left\{ 
\begin{array}{c}
S_{1}(N)=(1-\alpha )\sum_{j=0}^{N-1}\alpha ^{j}X_{N-j} \\ 
\\ 
S_{1}(N)=(1-\alpha )\sum_{j=0}^{N-1}j\alpha ^{j}X_{N-j}+(1-\alpha )S_{1}(N)%
\end{array}%
\right. 
\end{equation*}%
Qui donne le syst\`{e}me d'\'{e}quations lin\'{e}aire suivant%
\begin{equation*}
\begin{array}{c}
S_{1}(N)=\widehat{\alpha }_{1}-\widehat{\alpha }_{2}\frac{1}{1-\alpha }\text{
\ \ \ \ \ \ \ \ \ \ \ \ \ \ \ \ \ \ \ \ \ \ \ \ \ \ \ \ \ } \\ 
\\ 
S_{2}(N)-(1-\alpha )S_{1}(N)=\widehat{\alpha }_{1}\alpha -\widehat{\alpha }%
_{2}\frac{\alpha (\alpha +1)}{(1-\alpha )}%
\end{array}%
\end{equation*}%
On d\'{e}termine $\widehat{\alpha }_{1}$ et $\widehat{\alpha }_{2}$%
\begin{eqnarray*}
\widehat{\alpha }_{1} &=&2\cdot S_{1}(N)-S_{2}(N) \\
&& \\
\widehat{\alpha }_{2} &=&\frac{1}{1-\alpha }(S_{1}(N)-S_{2}(N))
\end{eqnarray*}

Cela implique 
\begin{eqnarray*}
S_{1}(N) &=&\alpha \cdot S_{1}(N-1)+(1-\alpha )\cdot X_{N} \\
&& \\
S_{2}(N) &=&\alpha \cdot S_{2}(N-1)+(1-\alpha )\cdot S_{1}(N)
\end{eqnarray*}%
Et les premiers valeurs de $S_{1}(N)$ et $S_{2}(N)$ sont $X_{1}$ et $%
(1-\alpha )^{2}X_{1}$ respevtivement.
\end{itemize}

\part{SUR LE PLAN PRATIQUE}

\subsection{\textbf{Implémentation des algorithmes en MATLAB} :}

\subsubsection{\protect\underline{Le code de lissage simple}:}

\_\_\_\_\_\_\_\_\_\_\_\_\_\_\_\_\_\_\_\_\_\_\_\_\_\_\_\_\_\_\_\_\_\_\_\_\_\_%
\_\_\_\_\_\_\_\_\_\_\_\_\_\_\_\_\_\_\_

\qquad \%code de lissage exponentiel simple

\qquad \%Faire entrer la s\'{e}rie chronologique X\_N et alpha

\qquad \%Etape 01 de lissage simple : Initialisation

\qquad \%La premi\`{e}re valeur de lissage

\qquad \qquad lissage\_simple(1)= X\_N(1);

\qquad \qquad lissage\_initial(1)=lissage\_simple(1);

\qquad\ \%Etape 02 de lissage simple : calcul de lissage

\qquad \qquad for i = 2:N+1

\qquad \qquad \qquad lissage\_simple(i) =
lissage\_initial(i-1)+(1-alpha)*(X\_N(i-1)-lissage\_initial(i-1));

\qquad \qquad \qquad lissage\_initial(i)=lissage\_simple(i);

\qquad \qquad end

\qquad\ \%Affichage des r\'{e}sultats

\qquad \qquad lissage\_simple;

\qquad \qquad lissage\_simple([1])=[]

\_\_\_\_\_\_\_\_\_\_\_\_\_\_\_\_\_\_\_\_\_\_\_\_\_\_\_\_\_\_\_\_\_\_\_\_\_\_%
\_\_\_\_\_\_\_\_\_\_\_\_\_\_\_\_\_\_\_

\subsubsection{\protect\underline{Le code de lissage double}:}

\_\_\_\_\_\_\_\_\_\_\_\_\_\_\_\_\_\_\_\_\_\_\_\_\_\_\_\_\_\_\_\_\_\_\_\_\_\_%
\_\_\_\_\_\_\_\_\_\_\_\_\_\_\_\_\_\_\_

\%Code de lissage exponentiel double

\qquad \%Faire entrer la s\'{e}rie chronologique X\_N et alpha et h

\qquad \%Etape 01 : initialisation de lissage

\qquad \qquad lissage\_double(1)= X\_N(1);

\qquad \qquad S\_1(1)=lissage\_double(1);

\qquad \qquad S\_2(1)=(1-alpha)\symbol{94}2*X\_N(2);

\qquad \%Etape 02 : calcul de lissage

\qquad \qquad for i = 2:N

\qquad \qquad \qquad \%le premier lissage

\qquad \qquad \qquad S\_1(i) = alpha*S\_1(i-1)+(1-alpha)*X\_N(i);

\qquad \qquad \qquad \%le deuxi\`{e}me lissage

\qquad \qquad \qquad S\_2(i) = alpha*S\_2(i-1)+(1-alpha)*S\_1(i);

\qquad \qquad end

\qquad \qquad\ S\_1

\qquad \qquad S\_2

\qquad\ \%Etape 03 : calcul des alpha's

\qquad \qquad alpha\_chap\_1(N) = 2*S\_1(N)-S\_2(N);

\qquad \qquad alpha\_chap\_1([1:N-1])=[]

\qquad \qquad alpha\_chap\_2(N) = ((1-alpha)/alpha)*(S\_1(N)-S\_2(N));

\qquad \qquad alpha\_chap\_2([1:N-1])=[]

\qquad\ \%Etape 04 : pr\'{e}vision pour la p\'{e}riode h

\qquad \qquad X\_chap\_(N+h) = alpha\_chap\_1+alpha\_chap\_2+h;

\qquad \qquad X\_chap\_(N+h)

\qquad \%Affichage des r\'{e}sultats

\qquad \qquad S\_1 = transpose(S\_1)

\qquad \qquad S\_2 = transpose(S\_2)

\qquad \qquad\ alpha\_chap\_1 = transpose(alpha\_chap\_1)

\qquad \qquad\ alpha\_chap\_2 = transpose(alpha\_chap\_2)

\qquad \qquad X\_chap = transpose(X\_chap)

\_\_\_\_\_\_\_\_\_\_\_\_\_\_\_\_\_\_\_\_\_\_\_\_\_\_\_\_\_\_\_\_\_\_\_\_\_\_%
\_\_\_\_\_\_\_\_\_\_\_\_\_\_\_\_\_\_\_

\subsubsection{\protect\underline{Code de g\'{e}n\'{e}ration des s\'{e}ries
chronologiques al\'{e}atoires}:}

\qquad La fonction suivante prend "d\_start" et "d\_end" comme des entr\'{e}%
es et sert \`{a} g\'{e}n\'{e}rer le temps :

\_\_\_\_\_\_\_\_\_\_\_\_\_\_\_\_\_\_\_\_\_\_\_\_\_\_\_\_\_\_\_\_\_\_\_\_\_\_%
\_\_\_\_\_\_\_\_\_\_\_\_\_\_\_\_\_\_\_

\qquad \%G\'{e}n\'{e}ration de temps pour les s\'{e}ries chronologique

\qquad \qquad function [ m ] = SerieChronologique(d\_start,d\_end)

\qquad \qquad \qquad \%La premi\`{e}re date de la s\'{e}rie

\qquad \qquad \qquad date\_start = datevec(d\_start);

\qquad \qquad \qquad \%La derni\`{e}re date de la s\'{e}rie

\qquad \qquad \qquad\ date\_end = datevec(d\_end);

\qquad \qquad \qquad tmp = (1:[date\_end-date\_start]*[12 1 0 0 0 0]')'-1;

\qquad \qquad \qquad u = ones(size(tmp));

\qquad \qquad \qquad m = datenum([date\_start(1)*u, date\_start(2)+tmp, u*[1
0 0 0]]);

\qquad \qquad end

\_\_\_\_\_\_\_\_\_\_\_\_\_\_\_\_\_\_\_\_\_\_\_\_\_\_\_\_\_\_\_\_\_\_\_\_\_\_%
\_\_\_\_\_\_\_\_\_\_\_\_\_\_\_\_\_\_\_

\qquad Puis, on ex\'{e}cute le code suivant pour avoir les donn\'{e}es d'une
s\'{e}rie chronologique sans tendance :

\_\_\_\_\_\_\_\_\_\_\_\_\_\_\_\_\_\_\_\_\_\_\_\_\_\_\_\_\_\_\_\_\_\_\_\_\_\_%
\_\_\_\_\_\_\_\_\_\_\_\_\_\_\_\_\_\_\_

\%G\'{e}n\'{e}ration : S\'{e}rie chronologoique avec tendance

\qquad \%G\'{e}n\'{e}ration : Temps

\qquad \%dd-mmm-d : format de la date entr\'{e}e

\qquad \%Choisir une date de d\'{e}but et une date de fin

\qquad \qquad date =
datestr(SerieChronologique('d\_start','d\_end'),'dd-mmm-yy');

\qquad\ \%G\'{e}n\'{e}ration : Donn\'{e}es

\qquad \qquad A = randi([1200 1950],24,1);

\qquad \qquad val = sort(A);

\qquad\ \%G\'{e}n\'{e}ration : S\'{e}rie chronologique

\qquad \qquad esp=' '; espace=repmat(esp,24,1);

\qquad \qquad time\_serie = strcat([date,espace,int2str(val)])

\_\_\_\_\_\_\_\_\_\_\_\_\_\_\_\_\_\_\_\_\_\_\_\_\_\_\_\_\_\_\_\_\_\_\_\_\_\_%
\_\_\_\_\_\_\_\_\_\_\_\_\_\_\_\_\_\_\_

\qquad Et on ex\'{e}cute le code suivant pour avoir les donn\'{e}es d'une s%
\'{e}rie chronologique avec tendance :

\_\_\_\_\_\_\_\_\_\_\_\_\_\_\_\_\_\_\_\_\_\_\_\_\_\_\_\_\_\_\_\_\_\_\_\_\_\_%
\_\_\_\_\_\_\_\_\_\_\_\_\_\_\_\_\_\_\_

\%G\'{e}n\'{e}ration : S\'{e}rie chronologoique avec tendance

\qquad \%G\'{e}n\'{e}ration : Temps

\qquad \%dd-mmm-d : format de la date entr\'{e}e

\qquad \%Choisir une date de d\'{e}but et une date de fin

\qquad \qquad date =
datestr(SerieChronologique('d\_start','d\_end'),'dd-mmm-yy');

\qquad\ \%G\'{e}n\'{e}ration : Donn\'{e}es

\qquad \qquad val = randi([1200 1950],24,1);

\qquad\ \%G\'{e}n\'{e}ration : S\'{e}rie chronologique

\qquad \qquad esp=' '; espace=repmat(esp,24,1);

\qquad \qquad time\_serie = strcat([date,espace,int2str(val)])

\_\_\_\_\_\_\_\_\_\_\_\_\_\_\_\_\_\_\_\_\_\_\_\_\_\_\_\_\_\_\_\_\_\_\_\_\_\_%
\_\_\_\_\_\_\_\_\_\_\_\_\_\_\_\_\_\_\_

\subsection{\textbf{Applications} :}

\subsubsection{\protect\underline{1\`{e}re application} : lissage
exponentiel simple}

\qquad Pour "$N=12$", "$d\_start=1-jan-2020$" et "$d\_end=1-jan-2021$", on
applique le code de pr\'{e}vision suivant pour :

\begin{itemize}
\item[•]G\'{e}n\'{e}rer une s\'{e}rie chronologique sans tendance.

\item[•] Choisir une meilleure valeur de la constante $\alpha .$

\item[•] Calculer le lissage simple.
\end{itemize}

\_\_\_\_\_\_\_\_\_\_\_\_\_\_\_\_\_\_\_\_\_\_\_\_\_\_\_\_\_\_\_\_\_\_\_\_\_\_%
\_\_\_\_\_\_\_\_\_\_\_\_\_\_\_\_\_\_\_

display('1\TEXTsymbol{\backslash} LISSAGE EXPONENTIEL SIMPLE : S\'{e}rie
chronologique sans tendance')

\%Donner une valeur \`{a} N : N d\'{e}signe le nombre des \'{e}lements de la
s\'{e}rie

\qquad N = 12 ;

\qquad \%G\'{e}n\'{e}ration : S\'{e}rie chronologoique sans tendance

\qquad \qquad \%G\'{e}n\'{e}ration : Temps

\qquad \qquad \%'dd-mm-yy' : la forme de la date entr\'{e}e

\qquad \qquad \%Choisir une date de d\'{e}but et une date de fin

\qquad \qquad \qquad date =
datestr(SerieChronologique('1-jan-2019','1-jan-2020'),'dd-mmm-yy');

\qquad \qquad \%G\'{e}n\'{e}ration : Donn\'{e}es

\qquad \qquad \qquad val = randi([950 2000],N,1);

\qquad \qquad \%Affichage : S\'{e}rie chronologique

\qquad \qquad \qquad esp = ' '; espace = repmat(esp,N,1);

\qquad \qquad \qquad serie\_chronologique =
strcat([date,espace,int2str(val)])

\qquad \%Lissage simple et le choix de la meilleure constante d'alpha

\qquad \qquad \%D\'{e}claration des variables

\qquad \qquad \qquad alpha = 0.01; j = 1;

\qquad \qquad \qquad X\_N = val;

\qquad \qquad \%Initialisation de min de l'erreur carr\'{e} et de la
meilleure constante

\qquad \qquad \%d'alpha

\qquad \qquad \qquad min\_Err\_carre = inf;

\qquad \qquad \qquad meilleur\_alpha = 0.01;

\qquad \qquad \%faire varier le alpha

\qquad \qquad \qquad while(alpha \TEXTsymbol{<}= 0.99)

\qquad \qquad \qquad \qquad \%Etape 01 de lissage simple : Initialisation

\qquad \qquad \qquad \qquad\ \%La premi\`{e}re valeur de lissage

\qquad \qquad \qquad \qquad lissage\_simple(1) = X\_N(1);

\qquad \qquad \qquad \qquad lissage\_initial(1) = lissage\_simple(1);

\qquad \qquad \qquad \qquad \%Etape 02 de lissage simple : calcul de lissage

\qquad \qquad \qquad \qquad for i = 2:N+1

\qquad \qquad \qquad \qquad \qquad \%Calcul : lissage simple

\qquad \qquad \qquad \qquad \qquad lissage\_simple(i) =
lissage\_initial(i-1)+(1-alpha)*(X\_N(i-1)-lissage\_initial(i-1));

\qquad \qquad \qquad \qquad \qquad lissage\_initial(i) = lissage\_simple(i);

\qquad \qquad \qquad \qquad \qquad \%Calcul : l'erreur

\qquad \qquad \qquad \qquad \qquad Err(i)=X\_N(i-1)-lissage\_simple(i);

\qquad \qquad \qquad \qquad \qquad \%Calcul : l'erreur carr\'{e}

\qquad \qquad \qquad \qquad \qquad Err\_carre(i) = Err(i)\symbol{94}2;

\qquad \qquad \qquad \qquad \qquad \%Calcul : la moyenne de l'erreur carr%
\'{e}

\qquad \qquad \qquad \qquad \qquad Moyenne\_Err\_carre(j) =
sum(Err\_carre)/N;

\qquad \qquad \qquad \qquad end

\qquad \qquad \qquad \qquad \%La condition pour que alpha soit meilleure

\qquad \qquad \qquad \qquad if (Moyenne\_Err\_carre(j)\TEXTsymbol{<}%
min\_Err\_carre)

\qquad \qquad \qquad \qquad \qquad min\_Err\_carre = Moyenne\_Err\_carre(j);

\qquad \qquad \qquad \qquad \qquad meilleur\_alpha = alpha;

\qquad \qquad \qquad \qquad end

\qquad \qquad \qquad \qquad \%Incr\'{e}mentation

\qquad \qquad \qquad \qquad alpha = alpha + 0.01;

\qquad \qquad \qquad \qquad\ j = j + 1;

\qquad \qquad \qquad \qquad lissage\_simple([1]) = [];

\qquad \qquad \qquad \qquad Err([1]) = [];

\qquad \qquad \qquad \qquad Err\_carre([1]) = [];

\qquad \qquad \qquad end

\qquad \qquad \%Affichage des r\'{e}sultats

\qquad \qquad \qquad lissage\_exponentiel\_simple =
transpose(lissage\_simple)

\qquad \qquad \qquad Erreur = transpose(Err)

\qquad \qquad \qquad Erreur\_carre = transpose(Err\_carre)

\qquad \qquad \qquad meilleur\_alpha

\_\_\_\_\_\_\_\_\_\_\_\_\_\_\_\_\_\_\_\_\_\_\_\_\_\_\_\_\_\_\_\_\_\_\_\_\_\_%
\_\_\_\_\_\_\_\_\_\_\_\_\_\_\_\_\_\_\_

\\

\qquad On aura les r\'{e}sultas suivants : \\
\begin{center}
\includegraphics[width=15cm]{ex_pre_liss_simple_1.png}
\end{center}\\
\begin{center}
\includegraphics[width=7cm]{ex_pre_liss_simple_2.png}\hspace{15mm}\includegraphics[width=3cm]{ex_pre_liss_simple_3.png}
\end{center}\\
\begin{center}

\includegraphics[width=5cm]{ex_pre_liss_simple_4.png}
\end{center}
\\
\begin{center}
\includegraphics[width=5cm]{ex_pre_liss_simple_5.png}
\end{center}


\subsubsection{\protect\underline{2\`{e}me application} : lissage
exponentiel double}

\qquad Pour "$N=24$", "$d\_start=1-jan-2020$" , "$d\_end=1-jan-2020$" et $%
h=9 $, on applique le code de pr\'{e}vision suivant pour :

\begin{itemize}
\item [•] G\'{e}n\'{e}rer une s\'{e}rie chronologique avec tendance.

\item[•] Choisir une meilleure valeur de la constante $\alpha .$

\item[•] Calculer le lissage double.
\end{itemize}

\bigskip

\ \ \ 

\_\_\_\_\_\_\_\_\_\_\_\_\_\_\_\_\_\_\_\_\_\_\_\_\_\_\_\_\_\_\_\_\_\_\_\_\_\_%
\_\_\_\_\_\_\_\_\_\_\_\_\_\_\_\_\_\_\_

display('1\TEXTsymbol{\backslash} LISSAGE EXPONENTIEL DOUBLE')

display(' S\'{e}rie chronologique avec tendance')

\qquad \%Choisir une valeur pour N et h

\qquad \qquad N = 24; h = 9;

\qquad \%G\'{e}n\'{e}ration : S\'{e}rie chronologoique avec tendance

\qquad \qquad \%G\'{e}n\'{e}ration : Temps

\qquad \qquad \%dd-mmm-d : format de la date entr\'{e}e

\qquad \qquad \%Choisir une date de d\'{e}but et une date de fin

\qquad \qquad \qquad date =
datestr(SerieChronologique('1-jan-2020','1-jan-2022'),'dd-mmm-yy');

\qquad \qquad \%G\'{e}n\'{e}ration : Donn\'{e}es

\qquad \qquad \qquad A = randi([950 2000],N,1);

\qquad \qquad \qquad val = sort(A);

\qquad \qquad \%Affichage : S\'{e}rie chronologique

\qquad \qquad \qquad esp = ' '; espace = repmat(esp,N,1);

\qquad \qquad \qquad serie\_chronologique =
strcat([date,espace,int2str(val)])

\qquad \%Lissage double et le choix de la meilleure constante d'alpha

\qquad \qquad \%D\'{e}claration des variables

\qquad \qquad \qquad alpha = 0.01; j = 1;

\qquad \qquad \qquad X\_N = val;

\qquad \qquad \%Initialisation de min de l'erreur carr\'{e} et de la
meilleure constante

\qquad\ \qquad \%d'alpha

\qquad \qquad \qquad min\_Err\_carre = inf;

\qquad \qquad \qquad meilleur\_alpha = 0.01;

\qquad \qquad \%faire varier le alpha

\qquad \qquad while(alpha \TEXTsymbol{<}= 0.99)

\qquad \qquad \%Etape 01 : initialisation de lissage

\qquad \qquad \qquad lissage\_double(1 )= X\_N(1);

\qquad \qquad \qquad S\_1(1) = lissage\_double(1);

\qquad \qquad S\_2(1) = (1-alpha)\symbol{94}2*X\_N(1);

\qquad \qquad \%Etape 02 : calcul de lissage

\qquad \qquad \qquad for i = 2:N

\qquad \qquad \qquad \qquad \%le premier lissage

\qquad \qquad \qquad \qquad S\_1(i) = alpha*S\_1(i-1)+(1-alpha)*X\_N(i);

\qquad \qquad \qquad \qquad \%le deuxi\`{e}me lissage

\qquad \qquad \qquad \qquad S\_2(i) = alpha*S\_2(i-1)+(1-alpha)*S\_1(i);

\qquad \qquad \qquad end

\qquad \qquad \%Etape 03 : calcul des alpha's

\qquad \qquad \qquad for i = 1:N

\qquad \qquad \qquad \qquad alpha\_chap\_1(i) = 2*S\_1(i)-S\_2(i);

\qquad \qquad \qquad \qquad alpha\_chap\_2(i) =
((1-alpha)/alpha)*(S\_1(i)-S\_2(i));

\qquad \qquad \qquad end

\qquad\ \qquad \%Etape 04 : pr\'{e}vision pour la p\'{e}riode h

\qquad \qquad \qquad for i = 1:N

\qquad \qquad \qquad \qquad X\_chap(i) =
alpha\_chap\_1(i)+(alpha\_chap\_2(i)*h);

\qquad\ \qquad \qquad \qquad \%Calcul : l'erreur

\qquad \qquad \qquad \qquad Err(i) = X\_N(i)-X\_chap(i);

\qquad \qquad \qquad \qquad \%Calcul : l'erreur carr\'{e}

\qquad \qquad \qquad \qquad Err\_carre(i) = Err(i)\symbol{94}2;

\qquad \qquad \qquad end

\qquad \qquad \qquad \%Calcul : la moyenne de l'erreur carr\'{e}

\qquad \qquad \qquad \qquad Moyenne\_Err\_carre(j) = sum(Err\_carre)/N;

\qquad \qquad \qquad \%La condition pour que alpha soit meilleure

\qquad \qquad \qquad \qquad if (Moyenne\_Err\_carre(j)\TEXTsymbol{<}%
min\_Err\_carre)

\qquad \qquad \qquad \qquad \qquad\ min\_Err\_carre = Moyenne\_Err\_carre(j);

\qquad\ \qquad \qquad \qquad \qquad meilleur\_alpha = alpha;

\qquad \qquad \qquad \qquad end

\qquad \qquad \qquad \%Incr\'{e}mentation d'alpha

\qquad \qquad \qquad \qquad alpha = alpha + 0.01;

\qquad \qquad \qquad \qquad j = j + 1;

\qquad \qquad end

\qquad\ \qquad \%Affichage des r\'{e}sultats

\qquad \qquad \qquad S\_1 = transpose(S\_1)

\qquad \qquad \qquad S\_2 = transpose(S\_2)

\qquad \qquad alpha\_chap\_1 = transpose(alpha\_chap\_1)

\qquad \qquad alpha\_chap\_2 = transpose(alpha\_chap\_2)

\qquad \qquad X\_chap = transpose(X\_chap)

\qquad \qquad Erreur = transpose(Err)

\qquad \qquad Erreur\_carre = transpose(Err\_carre)

\qquad \qquad meilleur\_alpha

\_\_\_\_\_\_\_\_\_\_\_\_\_\_\_\_\_\_\_\_\_\_\_\_\_\_\_\_\_\_\_\_\_\_\_\_\_\_%
\_\_\_\_\_\_\_\_\_\_\_\_\_\_\_\_\_\_\_

\\

\qquad On aura les r\'{e}sultats suivants :%
\\
\begin{center}
\includegraphics[width=10cm]{ex_pre_liss_double_1.png}
\end{center}\\
\begin{center}
\includegraphics[width=5cm]{ex_pre_liss_double_2.png}\hspace{5mm}\includegraphics[width=5cm]{ex_pre_liss_double_3.png}\hspace{5mm}\includegraphics[width=5cm]{ex_pre_liss_double_4.png}\hspace{5mm}\includegraphics[width=5cm]{ex_pre_liss_double_5.png}
\end{center}\\
\begin{center}

\includegraphics[width=5cm]{ex_pre_liss_double_6.png}\hspace{5mm}\includegraphics[width=5cm]{ex_pre_liss_double_7.png}\hspace{5mm}\includegraphics[width=5cm]{ex_pre_liss_double_8.png}
\end{center}
\\
\begin{center}
\includegraphics[width=5cm]{ex_pre_liss_double_9.png}
\end{center}





\end{document}